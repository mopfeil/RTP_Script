%************************************************************************
% Allgemeine Macros, insbesondere fr Skripten
%************************************************************************

\newlength{\qboxwidth}
\setlength{\qboxwidth}{0.89\textwidth}
\newcommand{\qboxcolor}{white}
\newcommand{\tboxcolor}{white}

\newcommand{\mycaption}[1]{\caption[]{\footnotesize #1}}
\usepackage{color}
\definecolor{hellgrau}{gray}{0.9}

\usepackage{listings}
\lstloadlanguages{C}
\lstset{language=C,commentstyle=\scriptsize, extendedchars=true} 

%---------------------------------------------------------------------- 
% Makro f"ur Floatingpoint-Zahlen 
\newcommand{\fl}{\mbox{fl}}

%---------------------------------------------------------------------- 
% Makro f"ur Mengenbuchstaben
%---------------------------------------------------------------------- 
\newcommand{\real}{{\mathbb{R}}}
%\newcommand{\r}{{\mathbb{R}}}
\newcommand{\realplus}{{\mathbb{R}}^+}
\newcommand{\z}{{\mathbb{Z}}}
\newcommand{\q}{{\mathbb{Q}}}
\newcommand{\C}{{\mathbb{C}}}
\newcommand{\nat}{{\mathbb{N}}}  
\newcommand{\n}{{\mathbb{N}}}
\newcommand{\myset}[1]{{\mathbb{#1}}}

\newcommand{\nnull}{{\mathbb{N}_0}}
\newcommand{\sigs}{\Sigma^{\ast}}

\newcommand{\rn}{\real^n}

\newcommand{\qed}{\mbox{}\hfill\mbox{$\box$}}
\newcommand{\pp}{\vspace{1.9ex}}                                          
\newcommand{\ind}[1]{#1}  %%% Index-Makro (Krypto-Buch) wird hier
                          %%% nicht verwendet

%\newcommand{\aufg}[1]{\subsubsection*{Aufgabe #1}}
%\newcommand{\teilaufg}[1]{\subsubsection*{#1)}}
%\newcommand{\teilaufg}[1]{\paragraph*{#1)}}
\newcommand{\blatt}[1]{\begin{center}\vspace*{5mm}\LARGE\bf \"Ubungsblatt #1\end{center}}
\newenvironment{alist}{\begin{list}{}{\labelsep 0mm \itemsep 0mm \parsep 0mm}}{\end{list}}
\newenvironment{blist}{\begin{list}{}{\labelsep 0mm \itemsep 0mm \parsep 2mm}}{\end{list}}

\newcommand{\abc}[1]{\item[\bf #1)\hspace*{4mm}]}
\renewcommand{\d}[2]{{\partial #1 \over \partial #2}}
\newcommand{\vect}[1]{\left( \begin{array}{c} #1 \end{array} \right)}
\newcommand{\matr}[2]{\left( \begin{array}{#1} #2 \end{array} \right)}
\newcommand{\bincoeff}[2]{\left(\rule{0mm}{0.5cm}\right.\!\!\!\!\begin{array}{c} #1 \\[-0.5ex] #2 \end{array} \!\!\!\!\left.\rule{0mm}{0.5cm} \right) }
%---------------------------------------------------------------------- 
% Makro f"ur 2x2 Matrix mit ()
%---------------------------------------------------------------------- 
\newcommand{\matrzwei}[4]{\left(\begin{array}{cc}#1&#2\\#3&#4\\\end{array}\right)}

\newcommand{\degr}{\mbox{deg}}
\newcommand{\lbr}[1]{\left\{ \begin{array}{cl} #1 \end{array} \right.}
\newcommand{\grad}{\mbox{grad}}
\newcommand{\rotation}{\mbox{rot}}
\renewcommand{\div}{\mbox{div}}
\newcommand{\condfunct}[5]{#1 = \left\{\begin{array}{cl}%
                           #2 & \mbox{falls } #3\\%
                           #4 & \mbox{falls } #5%
             \end{array}%
        \right.}
\newcommand{\condfunctsonst}[4]{#1 = \left\{\begin{array}{cl}%
                           #2 & \mbox{falls } #3\\%
                           #4 & \mbox{sonst }%
             \end{array}%
        \right.}
\newcommand{\condfunctiii}[7]{#1 = \left\{\begin{array}{cl}%
                           #2 & \mbox{falls } #3\\%
                           #4 & \mbox{falls } #5\\%
                           #6 & \mbox{falls } #7%
             \end{array}%
        \right.}
\newcommand{\bm}[1]{\mbox{\boldmath $\displaystyle #1$}}
\newcommand{\lr}[1]{\left( #1 \right)}
\newcommand{\dfb}{differenzierbar }

\newcommand{\eps}{\varepsilon}
\newcommand{\aequi}{\quad\Leftrightarrow\quad}
\newcommand{\gdw}{\aequi}
\newcommand{\impl}{\quad\Rightarrow\quad}
\newcommand{\bem}[1]{{\bf Bemerkung: }#1}
\newcommand{\halb}{\frac{1}{2}}


%------------------------------------------------------------------------------
%       Aufgaben fr �ungsbl�ter
%------------------------------------------------------------------------------

\newcounter{AufgNr}
\setcounter{AufgNr}{0}

\newcounter{TeilaufgNr}


\newcommand{\aufg}%%%%%%%%%%%% Aufgabe %%%%%%%%%%%%
        {\refstepcounter{AufgNr}
         \subsubsection*{Aufgabe \arabic{AufgNr}}%
         \setcounter{TeilaufgNr}{1}%
        }
%----------------------------------------------------------------------
% Env fr Aufgabe
%----------------------------------------------------------------------
\newenvironment{aufgenv}
        {\refstepcounter{AufgNr}
         \subsubsection*{Aufgabe \arabic{AufgNr}}%
         \setcounter{TeilaufgNr}{1}%
\vspace*{-1ex}
        }{}
\newcommand{\klaufg}[1]%%%%%%%%%%%% Aufgabe %%%%%%%%%%%%
        {\refstepcounter{AufgNr}
         \subsubsection*{Aufgabe \arabic{AufgNr} ~~~\mbox{\rm\footnotesize (#1)}}%
         \setcounter{TeilaufgNr}{1}%
        }
\newcommand{\ta}%%%%%%%%%%%%% Teilaufgabe %%%%%%%%%%%%
        {\item[\bf \alph{TeilaufgNr})\hspace*{4mm}]%
        \refstepcounter{TeilaufgNr}%
        }
\newenvironment{talist}{\begin{list}{}%%%%%%%%%% Liste von Teilaufgaben %%%%%%
        {\labelsep 0mm \itemsep 0mm \parsep 2mm}}{\end{list}}

\newcommand{\lsg}[1]{\subsubsection*{L�ung zu Aufgabe \arabic{AufgNr}} \setcounter{TeilaufgNr}{1} #1}
%\renewcommand{\lsg}[1]{}
\newcommand{\keinelsg}{}

%------------------------------------------------------------------------------


%---------------------------------------------------------------------- 
% Makro zur Erzeugung eigener Bilder (ohne Figure Env.)
%---------------------------------------------------------------------- 
\newcommand{\mypic}[3]{\begin{center}
                                \rule{0mm}{0mm}
                                \epsfig{file=figures/#1,width=#2cm,height=#3cm}
                            \end{center}}
\newcommand{\pic}[2]{\epsfig{file=#1,width=#2\textwidth}}
\newcommand{\picc}[2]{\begin{center}\includegraphics[width=#2\textwidth]{#1}\end{center}}
\newcommand{\mypict}[1]{\begin{center}
                                \rule{0mm}{0mm}
                                \epsfig{#1}
                            \end{center}}
%---------------------------------------------------------------------- 
% Makros zur Erzeugung eigener Abbildungen (mit Figure Env.)
%---------------------------------------------------------------------- 
\newcommand{\myfigure}[5]{\begin{figure}[htbp]
                            \begin{center}
                                \rule{0mm}{0mm}
                                \epsfig{file=figures/#2,width=#3cm,height=#4cm}
                            \end{center}
                            \mycaption{#5. \label{#1}}
                          \end{figure}}
\newcommand{\mypsfigure}[3]{\begin{figure}[htbp]
                            \begin{center}
                                \rule{0mm}{0mm}
                                \epsfig{#2}
                            \end{center}
                            \mycaption{#3. \label{#1}}
                          \end{figure}}

\newcommand{\myfig}[3]{\begin{figure}[htbp]
                            \begin{center}
                                \rule{0mm}{0mm}
                                \epsfig{file=figures/#2}
                            \end{center}
                            \mycaption{#3. \label{#1}}
                          \end{figure}}

%---------------------------------------------------------------------- 
% Makro zum Verweis auf eigene Abbildungen
%---------------------------------------------------------------------- 
\newcommand{\abb}[1]{Abb.~\ref{fig:#1}}

%---------------------------------------------------------------------- 
% Makro fr Problem
%---------------------------------------------------------------------- 
\newcounter{problem}
\newtheorem{problem}{Problem}
\newcommand{\prob}[1]{\begin{problem}
                        {\rm #1}
                      \end{problem}}

%---------------------------------------------------------------------- 
% Makro fr L�ung
%---------------------------------------------------------------------- 
\newcounter{solution}
\newtheorem{solution}{L�ung}
\newcommand{\sol}[1]{\begin{solution}
                        {\rm #1}
                      \end{solution}}

%---------------------------------------------------------------------- 
% Makro fr Beispiel
%---------------------------------------------------------------------- 
\newcounter{example}
\newtheorem{example}{Beispiel}[chapter]
\newcommand{\exa}[1]{\begin{example}
                        {\rm #1}
%                       \vspace*{3ex}
                      \end{example}}

%---------------------------------------------------------------------- 
% Makro fr umrandetes Corollar
%---------------------------------------------------------------------- 
\newcounter{corollar}
\newtheorem{corollar}{Corollar}[section]
\newcommand{\coro}[2]{\begin{corollar}
                        \rm #2
                        \label{def:#1}
                      \end{corollar}}

%---------------------------------------------------------------------- 
% Makro fr Lemma
%---------------------------------------------------------------------- 
\newcounter{mylemma}
\newtheorem{mylemma}{Lemma}[chapter]
\newcommand{\lemma}[2]{\begin{mylemma}
                        \rm #2
                        \label{lem:#1}
                      \end{mylemma}}

%---------------------------------------------------------------------- 
% Makro fr Satz
%---------------------------------------------------------------------- 
\newcounter{mysatz}
\newtheorem{mysatz}{Satz}[chapter]
\newcommand{\satz}[2]{\begin{mysatz}
                        \rm #2
                        \label{satz:#1}
                      \end{mysatz}}

%---------------------------------------------------------------------- 
% Makro fr umrandete Definitionen
%---------------------------------------------------------------------- 
\newcounter{definition}
\newtheorem{mydefn}{Definition}[chapter]
\newcommand{\defn}[2]{\vspace*{4mm}
%% \refstepcounter{definition}
\noindent 
$\displaystyle \left\|
                \mbox{
                \begin{minipage}[c]{1.1\qboxwidth} \vspace*{0.4mm}
        \begin{mydefn}
                \rm #2
                \label{def:#1}
        \end{mydefn}
                \end{minipage} }  \right. $\medskip}
%---------------------------------------------------------------------- 
% Makro fr Deflist
%---------------------------------------------------------------------- 
\newenvironment{deflist}{\vspace*{1ex}\begin{list}{}
   {\topsep 0em \itemindent -0.5em \leftmargin 1em \itemsep 0em \parsep 0.3em}
}{\end{list}}
\newcommand{\defit}{\item[$\bullet$]}

%---------------------------------------------------------------------- 
% Makros fr Beweis
%---------------------------------------------------------------------- 
\newenvironment{beweis}{%\vspace*{-3ex}
\noindent{\bf Beweis: }}{\hfill\raisebox{-.1ex}{$\Box$}\\}
\newenvironment{beweisobox}{\noindent{\bf Beweis: }}{}

%% %---------------------------------------------------------------------- 
%% % Makro fr Beispiel
%% %---------------------------------------------------------------------- 
%% \newcounter{example}
%% \newtheorem{example}{Beispiel}[section]
%% \newcommand{\exa}[1]{\begin{example}
%%                         {\rm #1}
%%                         \vspace*{3ex}
%%                       \end{example}}
%% 
%% %---------------------------------------------------------------------- 
%% % Makro fr umrandetes Corollar
%% %---------------------------------------------------------------------- 
%% \newcounter{corollar}
%% \newtheorem{corollar}{Corollar}
%% \newcommand{\coro}[2]{\begin{corollar}
%%                         #2
%%                         \label{cor:#1}
%%                       \end{corollar}}
%% 
%% %---------------------------------------------------------------------- 
%% % Makro fr Lemma
%% %---------------------------------------------------------------------- 
%% \newcounter{mylemma}
%% \newtheorem{mylemma}{Lemma}
%% \newcommand{\lemma}[2]{\begin{mylemma}
%%                         #2
%%                         \label{lem:#1}
%%                       \end{mylemma}}
%% 
%% %---------------------------------------------------------------------- 
%% % Makro fr Satz
%% %---------------------------------------------------------------------- 
%% \newcounter{mysatz}
%% \newtheorem{mysatz}{Satz}
%% \newcommand{\satz}[2]{\fbox{\parbox{\textwidth}{\begin{mysatz}
%%                         #2
%%                         \label{satz:#1}
%%                       \end{mysatz}}}\vspace*{4mm}
%% }
%% 
%% %---------------------------------------------------------------------- 
%% % Makros fr Beweis
%% %---------------------------------------------------------------------- 
%% \newenvironment{beweis}{\vspace*{-3ex}\noindent{\bf Beweis: }}{\hfill\raisebox{-.1ex}{$\Box$}\\}
%% 
%% %% %---------------------------------------------------------------------- 
%% %% % Makro fr umrandete Definitionen
%% %% %---------------------------------------------------------------------- 
%% \newcounter{definition}
%% \newtheorem{definition}{Definition}
%% \newcommand{\defn}[2]{\vspace*{4mm}
%% \noindent 
%% $\displaystyle \left\|
%%                 \mbox{
%%                 \begin{minipage}[c]{15.8cm} 
%%                 \begin{definition}
%%                         #2
%%                         \label{def:#1}
%%                  \end{definition}
%%                 \end{minipage} }  \right. $ \\[2ex] }
%% 
%% %---------------------------------------------------------------------- 
%% % Makro fr Deflist
%% %---------------------------------------------------------------------- 
%% \newenvironment{deflist}{\vspace*{1ex}\begin{list}{}
%%    {\topsep 0em \itemindent -0.5em \leftmargin 1em \itemsep 0em \parsep 0.3em}
%% }{\end{list}}
%% \newcommand{\defit}{\item[$\bullet$]}


%%%%%%%%% eingerahmter Text %%%%%%
\newcommand{\qbox}[2][0.9]{
   \begin{center}
        \fcolorbox{black}{\qboxcolor}{\parbox{#1\qboxwidth}{#2}}
   \end{center}}

%%%%%%%%% eingerahmter Text, ganze Breite %%%%%%
\newcommand{\tbox}[1]{
   \begin{center}
        \fcolorbox{black}{\tboxcolor}{\parbox{\textwidth}{#1}}
   \end{center}}


%%%%%%%%% fr Programmtexte, siehe a. Ginf2 %%%%%%
\newcommand{\programmlist}[1]{\begin{quote}\colorbox{\tboxcolor}{\begin{tabular}{|@{\tt~}l|}\hline #1 \hline\end{tabular}}\end{quote}}
\newenvironment{programm}{\begin{quote}\begin{tabular}{|@{\tt~}l|}\hline}
                         {\hline\end{tabular}\end{quote}}
\newcommand{\programminput}[3][1]{%
\sboxsep 0pt
\sdim 2pt
\centerline{\shabox{\colorbox{hellgrau}{~~~~\parbox{0.9\textwidth}{
\listinginput[#1]{#2}{#3}
}~~~~}}}}
% Aufruf: \programminput[Inkrementzeilnr]{Startzeile}{Dateiname}


%%%%%%%%% zwei Parboxen nebeneinander %%%%%%
\newcommand{\hsplit}[4]{\parbox{#1\textwidth}{#2}\hfill\parbox{#3\textwidth}{#4}}
\newcommand{\hsplitiii}[6]{\parbox[t]{#1\textwidth}{#2}\hfill\parbox[t]{#3\textwidth}{#4}\hfill\parbox[t]{#5\textwidth}{#6}}
\newcommand{\hsplitiiii}[8]{\parbox[t]{#1\textwidth}{#2}\hfill\parbox[t]{#3\textwidth}{#4}\hfill\parbox[t]{#5\textwidth}{#6}\hfill\parbox[t]{#7\textwidth}{#8}}
\newcommand{\hsplitiiig}[9]{\parbox[#2]{#1\textwidth}{#3}\hfill\parbox[#5]{#4\textwidth}{#6}\hfill\parbox[#8]{#7\textwidth}{#9}}
\newcommand{\hsplitiiio}[6]{\parbox{#1\textwidth}{#2}\hfill\parbox{#3\textwidth}{#4}\hfill\parbox{#5\textwidth}{#6}}
\newcommand{\hsplitt}[4]{\parbox[t]{#1\textwidth}{#2}\hfill\parbox[t]{#3\textwidth}{#4}}
\newcommand{\hsplitg}[6]{\parbox[#2]{#1\textwidth}{#3}\hfill\parbox[#5]{#4\textwidth}{#6}}
\newcommand{\hsplito}[4]{\parbox{#1\textwidth}{#2}\parbox{#3\textwidth}{#4}}
\newcommand{\hsplitfig}[4]{\parbox{#1\textwidth}{#2}\hfill\parbox{#3\textwidth}{\epsfig{file=#4,width=#3\textwidth}}}
\newcommand{\hsplitfigt}[4]{\parbox[t]{#1\textwidth}{#2}\hfill\parbox{#3\textwidth}{\parbox[t]{#3\textwidth}{\epsfig{file=#4,width=#3\textwidth}}}}
\newcommand{\hsplitzweifig}[4]{\parbox{#1\textwidth}{\epsfig{file=#2,width=#1\textwidth}}\hfill\parbox{#3\textwidth}{\epsfig{file=#4,width=#3\textwidth}}}
\newcommand{\hsplitdreifig}[6]{\parbox{#1\textwidth}{\epsfig{file=#2,width=#1\textwidth}}\hfill\parbox{#3\textwidth}{\epsfig{file=#4,width=#3\textwidth}}\hfill\parbox{#5\textwidth}{\epsfig{file=#6,width=#5\textwidth}}}
\newcommand{\hsplitfigo}[5]{\parbox{#1\textwidth}{#2}{\hspace*{#3\textwidth}}\parbox{#4\textwidth}{\epsfig{file=#5,width=#4\textwidth}}}

%%%%%%%%% aus DASI %%%%%%%%%%%%%%%%%%%%%%%%
%\newcommand{\mod}{\bmod}
\newcommand{\ggt}{\mathrm{ggT}}

%----------------------------------------------------------------------------
%----- bbo zeichnet einen Knoten und zwei Nachfolger
%----- 1. Arg.: Label, 2. Arg.: linker Unterbaum, 3. Arg.: rechter Unterbaum
\newcommand{\bbo}[3]{\begin{bundle}{#1} \chunk{#2} \chunk{#3}\end{bundle}}
%----- tbo zeichnet einen Knoten und drei Nachfolger
\newcommand{\tbo}[4]{\begin{bundle}{#1} \chunk{#2} \chunk{#3} \chunk{#4}\end{bundle}}
%----- ubo zeichnet einen Knoten und einen Nachfolger
\newcommand{\ubo}[2]{\begin{bundle}{#1} \chunk{#2} \end{bundle}}
%----- blo zeichnet einen Blattknoten
\newcommand{\blo}[1]{#1}

%----------------------------------------------------------------------------
%----- bb zeichnet einen Knoten und zwei Nachfolger mit Ovalbox
%----- 1. Arg.: Label, 2. Arg.: linker Unterbaum, 3. Arg.: rechter Unterbaum
\newcommand{\bb}[3]{\begin{bundle}{\ovalbox{#1}} \chunk{#2} \chunk{#3}\end{bundle}}
%----- ub zeichnet einen Knoten und einen Nachfolger mit Ovalbox
\newcommand{\ub}[2]{\begin{bundle}{\ovalbox{#1}} \chunk{#2} \end{bundle}}
%----- bl zeichnet einen Blattknoten mit Ovalbox
\newcommand{\bl}[1]{\ovalbox{#1}}

%----------------------------------------------------------------------------
%----- bbi zeichnet einen Knoten mit Index und zwei Nachfolger
%----- 1. Arg.: Label, 2. Arg.: Index, 3.Arg.: li. Unterb., 4.Arg.: re. Unterb.
\newcommand{\bbi}[4]{\begin{bundle}{$\,\,\ovalbox{#1}^{#2}$} \chunk{#3} \chunk{#4}\end{bundle}}
%----- ubi zeichnet einen Knoten mit Index und einen Nachfolger
\newcommand{\ubi}[3]{\begin{bundle}{$\,\,\ovalbox{#1}^{#2}$} \chunk{#3}\end{bundle}}
%----- bli zeichnet einen Blattknoten mit Index
\newcommand{\bli}[2]{$\,\,\ovalbox{#1}^{#2}$}

%%%%%%%%%%%%%%%%%% fr formale Sprachen %%%%%%%%%%%%%%%%%%%%%%%%%%%%%%

\newcommand{\va}[1]{$<\! \mbox{\it#1} \!>$}
\newcommand{\var}[1]{<\! \mbox{\it#1} \!>}
\renewcommand{\r}{\rightarrow}
\newcommand{\R}{\Rightarrow}


%%%%%%%%%%%%%%%%%% Macros v. Hr. Buck %%%%%%%%%%%%%%%%%%%%%%%%%%%%%%%%

\newcommand{\un}[1]{{\bfseries #1}}
%\newcommand{\rb}[1]{\raisebox{1.5ex}[-1.5ex]{#1}}
\newcommand{\rb}[1]{#1}
\newcommand{\mc}[1]{\multicolumn{1}{l|}{#1}}
\newcommand{\mcc}[1]{\multicolumn{1}{c|}{#1}}                                  
%\newcommand{\muc}[1]{\multicolumn{1}{l}{#1}}


\newcommand{\engl}[1]{(engl.\ #1)}

%---------------------------------------------------------------------- 
% Itemize
%---------------------------------------------------------------------- 
\newcommand{\ite}{\item[$\bullet$~~]}
\newcommand{\itee}{\item[$\circ$~~]}
\newcommand{\iteee}{\item[$\ast$~~]}
\newcommand{\bi}{\begin{itemize}}
\newcommand{\ei}{\end{itemize}}
\newcommand{\itd}[1]{\item[#1]}
\newcommand{\bd}{\begin{description}}
\newcommand{\ed}{\end{description}}

%*************** Euro Macros *****************************

\newcommand\eur{{\sffamily C%
    \makebox[0pt][l]{\kern-.70em\mbox{--}}%
    \makebox[0pt][l]{\kern-.68em\raisebox{.25ex}{--}}}}



%**************************************************************************
% Macros die nur auf Folien etwas bewirken
%**************************************************************************

%%%%% item mit grnem Punkt nur fr Folien (im Skript unsichtbar)
\newcommand{\itef}{}
\newcommand{\iteef}{}
\newcommand{\bif}{}
\newcommand{\eif}{}
\newcommand{\itdf}[1][]{}
\newcommand{\bdf}{}
\newcommand{\edf}{}
\newcommand{\np}{}

%%%%% Ausblenden von Text fr Folien
\newcommand{\nsl}[1]{#1} % text nur im Skript (nicht auf Folien)
\newcommand{\nl}{}      % Zeilenumbruch nur auf Folien
\newcommand{\os}[1]{}   % text nur auf Folien
\newcommand{\bff}{}      % Bold nur auf Folien

%%%%%%%%%%%%%%%%%%%%%%%%%%%%%%%%%%%%%%%%%%%%%%%%%%%%%%%%%%%%%%%%%%%%%%%%%%%%%%
%               PDF-Latex Demos
%%%%%%%%%%%%%%%%%%%%%%%%%%%%%%%%%%%%%%%%%%%%%%%%%%%%%%%%%%%%%%%%%%%%%%%%%%%%%%
\newcommand{\pdflaunch}[1] {\pdfpageattr{/AA << /O << /S /Launch /F (#1) >>>>}}
\newcommand{\pdflaunchlink}[2]{%
        \pdfstartlink attr{/Border [0 0 0]} user{/Subtype /Link /A << %
        /S /Launch /F (#1) >>}%
        \pdfliteral{0 1 0 0 k}%
        {#2}\pdfliteral{0 0 0 1 k}\pdfendlink%
        }
\newcommand{\demo}[2]{Demo: #2}
\newcommand{\demoneu}[2]{Demo: #2}

%%%%%%%%%%%%%%%%%%%%%%%%%%%%%%%%%%%%%%%%%%%%%%%%%%%%%%%%%%%%%%%%%%%%%%%%%%%%%%
\newcommand{\fehltnoch}[1]{\bigskip\fbox{\parbox{\textwidth}{\typeout{!!!!!!!!!!!!!!!!!!!!!!!! fehltnoch: #1}\sf {\bfseries
!!!!!!!!!!!!!!!!!!!!!!!!!!!!!!!!!!!!!!!!!!!!!!!!!!!!!\\Fehlt noch:}\\ #1\\{\bfseries
!!!!!!!!!!!!!!!!!!!!!!!!!!!!!!!!!!!!!!!!!!!!!!!!!!!!!}}}\bigskip}


%%%%%%%%%%%%%%%%%%%%%%%%%%%%%%%%%%%%%%%%%%%%%%%%%%%%%%%%%%%%%%%%%%%%%%%%%%%%%%
%               Makros fr Sonderzeichen
%%%%%%%%%%%%%%%%%%%%%%%%%%%%%%%%%%%%%%%%%%%%%%%%%%%%%%%%%%%%%%%%%%%%%%%%%%%%%%

\newcommand{\pr}{\begin{math}
 \rightarrow
\end{math} }


%%% Local Variables: 
%%% mode: latex
%%% TeX-master: t
%%% End: 
