%************************************************************************
% Spezielle Macros f�r Folien
% - viele Macros (renewcommand) �berschreiben Macros aus ./macros.tex
% - daher zuerst: input macros.tex
%         danach: input macros-sl.tex
%************************************************************************


%%%%%%%%%%%%%%%%%%%%%%%%%%%%%%%%%%%%%%%%%%%%%%%%%%%%%%%%%%%%%%%%%%%%%%%%%
%%% Spezielle Defs f"ur Folien

\renewcommand{\nsl}[1]{}
\renewcommand{\nl}{\\ \bigskip {\color{green} \rule{40mm}{0.5mm}}\\ \bigskip}
%\renewcommand{\ind}[1]{\color{red}\bfseries #1}
\renewcommand{\bff}{\bfseries\color{red}  } % auf Folien bold und rot
\renewcommand{\os}[1]{#1}                     % Text der nur auf Folien erscheint

\renewcommand{\np}{\newpage}


%%%%%%%%%%%%%%%%%%%% Farbdefinitionen %%%%%%%%%%%%%%%%%%%%%%%%%%%%%%%%
\newcommand{\red}{\color{red} }
\definecolor{hellgelb}{cmyk}{0.05,0,0.4,0}
\definecolor{dunkelgelb}{cmyk}{0.05,0.1,1,0.1}
\definecolor{schwgelb}{cmyk}{0,0,1,0.1}
\definecolor{headcol}{rgb}{0,0.3,0.75}
\definecolor{blau}{rgb}{0,0,1}
\definecolor{rot}{rgb}{1,0,0}
\definecolor{hellblau}{rgb}{0.85,0.85,1}
\definecolor{hellgruen}{rgb}{0.9,1,0.9}
\definecolor{gruen}{rgb}{0.4,0.8,0.5}
\renewcommand{\qboxcolor}{yellow}
\newcommand{\pagecol}{hellgelb}
\newcommand{\qboxframecolor}{schwgelb}
\setlength{\qboxwidth}{0.98\textwidth}
\setlength{\fboxrule}{0.7mm}
%\renewcommand{\bf}{\bfseries\color{magenta}  }
\renewcommand{\bf}{{\color{red}\bfseries }}
\renewcommand{\un}{{\color{red}\bfseries }}
\renewcommand{\em}{\bfseries\color{gruen} }

%%%%%%%%% eingerahmter Text %%%%%%
\renewcommand{\tboxcolor}{hellgruen}

%%%%%%% Makro f"ur umrandete Definitionen %%%%%
\renewcommand{\defn}[2]{\noindent\fcolorbox{dunkelgelb}{yellow}{\parbox{\qboxwidth
}{              \vspace*{-3mm}
                \begin{mydefn}
                        #2      
                        \label{def:#1}
                 \end{mydefn}
                \vspace*{-4mm}}}
                \vspace*{8mm}}
%%%%%%%%%%%%%%%%%%%%%%%%%%%%%%%%%%%%%%%%%%%%%%%%%%%%%%%%%%%%%%%%%%%%%%%%%

%---------------------------------------------------------------------- 
% Makro f"ur umrandetes Corollar
%---------------------------------------------------------------------- 
\renewcommand{\coro}[2]{\textcolor{blue}{\begin{corollar}
                        #2
                        \label{cor:#1}
                      \end{corollar}}}

%---------------------------------------------------------------------- 
% Makro f"ur Lemma
%---------------------------------------------------------------------- 
\renewcommand{\lemma}[2]{\textcolor{blue}{\begin{mylemma}
                        #2
                        \label{lem:#1}
                      \end{mylemma}}}

%---------------------------------------------------------------------- 
% Makro f"ur Satz
%---------------------------------------------------------------------- 
\renewcommand{\satz}[2]{\noindent\fcolorbox{blau}{hellblau}{\parbox{\qboxwidth
}{              \vspace*{-3mm}
                        \begin{mysatz}
                        #2
                        \label{satz:#1}
                      \end{mysatz}
                \vspace*{-4mm}}}
                \vspace*{4mm}}

%---------------------------------------------------------------------- 
% Makro f�r Beispiel
%---------------------------------------------------------------------- 
%\newcounter{example}
%\newtheorem{example}{Beispiel}[chapter]
\renewcommand{\exa}[1]{
\noindent\fcolorbox{hellgruen}{hellgruen}{\parbox{\textwidth}{\vspace*{-3mm}
                \begin{example}
                        \sf #1
                \end{example}
                \vspace*{-2ex}}}\\[2ex]
}
%\newcommand{\longexa}[1]{
\renewcommand{\exa}[1]{
{\color{headcol}
                \begin{example}
                        \sf #1
                \end{example}
}
}

%%%%%%%%% f�r Programmtexte, siehe a. Ginf2 %%%%%%
\renewcommand{\programmlist}[1]{\begin{quote}\colorbox{\tboxcolor}{\begin{tabular}{|@{\tt~}l|}\hline #1 \hline\end{tabular}}\end{quote}}

%---------------------------------------------------------------------- 
% Itemize
%---------------------------------------------------------------------- 
%%%%% item mit gr�nem Punkt
\renewcommand{\ite}{\item[\textcolor{headcol}{$\bullet$~}]}
\renewcommand{\itee}{\item[\textcolor{headcol}{$\circ$~}]}
\renewcommand{\bi}{\begin{itemize}}
\renewcommand{\ei}{\end{itemize}}
\renewcommand{\itd}[1]{\item[\textcolor{headcol}{\bfseries #1~}]}
\renewcommand{\bd}{\begin{description}}
\renewcommand{\ed}{\end{description}}
%%%%% item mit gr�nem Punkt nur f�r Folien (im Skript unsichtbar)
\renewcommand{\itef}{\item[\textcolor{headcol}{$\bullet$~}]}
\renewcommand{\iteef}{\item[\textcolor{headcol}{$\circ$~}]}
\renewcommand{\bif}{\begin{itemize}}
\renewcommand{\eif}{\end{itemize}}
\renewcommand{\itdf}[1][]{\item[\textcolor{headcol}{#1~}]}
\renewcommand{\bdf}{\begin{description}}
\renewcommand{\edf}{\end{description}}

%\input{amssym.def} \input{amssym.tex} % AMSTEX
% ==========================================================================
%                                               Globale Deklarationen
% ==========================================================================

%\renewcommand{\underline}[1]{\noindent{\bf #1}}

%---------------------------------------------------------------------- 
% damit die Underfull-Meldungen nicht mehr auftauchen
%---------------------------------------------------------------------- 
\hbadness = 10000
\vbadness = 10000

%---------------------------------------------------------------------- 
% damit gleitobjekte auf der richtigen Seite erscheinen
%---------------------------------------------------------------------- 
\setcounter{totalnumber}{10}
\setcounter{topnumber}{10}
\setcounter{bottomnumber}{10}
\renewcommand{\textfraction}{0.1}
\renewcommand{\topfraction}{0.9}
\renewcommand{\bottomfraction}{0.9}
\renewcommand{\floatpagefraction}{0.9}

%\renewcommand{\baselinestretch}{1.2}

\renewcommand{\familydefault}{cmss}

%%%%%%%%%%%%%%%%%%%%%%%%%%%%%%%%%%%%%%%%%%%%%%%%%%%%%%%%%%%%%%%%%%%%%%%%%%%%%%
%               HEADINGS
%%%%%%%%%%%%%%%%%%%%%%%%%%%%%%%%%%%%%%%%%%%%%%%%%%%%%%%%%%%%%%%%%%%%%%%%%%%%%%

%\headsep 5mm
%\pagestyle{fancyplain}
% \renewcommand{\chaptermark}[1]{\markboth{#1}{#1}} % remember chapter title
% \renewcommand{\chaptermark}[1]{\markboth{\thechapter~~#1}{}} % remember chaptertitle
% \renewcommand{\sectionmark}[1]{\markright{\thesection~~#1}}
                                                   % section number and title
% \lhead[\fancyplain{}{\bfseries\thepage}]{\fancyplain{}{\rightmark}}
% \rhead[\fancyplain{}{\leftmark}]{\fancyplain{}{\bfseries\thepage}}
%\footskip 8mm
%\cfoot{\rule{\textwidth}{0.1mm}\\\tiny Copyright \copyright  ~2001, W. Ertel}

%%%%%%%%%%%%%%%%%%%%%%%%%%%%%%%%%%%%%%%%%%%%%%%%%%%%%%%%%%%%%%%%%%%%%%%%%%%%%%
%               PDF-Latex Demos
%%%%%%%%%%%%%%%%%%%%%%%%%%%%%%%%%%%%%%%%%%%%%%%%%%%%%%%%%%%%%%%%%%%%%%%%%%%%%%

\renewcommand{\demo}[2]{\fcolorbox{gruen}{hellgruen}{Demo:\pdflaunchlink{#1}{#2}}}
\renewcommand{\demoneu}[2]{\fcolorbox{gruen}{hellgruen}{Demo:\href{run:#1}{#2}}}
%%%%%%%%%%%%%%%%%%%%%%%%%%%%%%%%%%%%%%%%%%%%%%%%%%%%%%%%%%%%%%%%%%%%%%%%%%%%%%
%               �berschriften
%%%%%%%%%%%%%%%%%%%%%%%%%%%%%%%%%%%%%%%%%%%%%%%%%%%%%%%%%%%%%%%%%%%%%%%%%%%%%%

% \renewcommand{\chapter}[1]{\refstepcounter{chapter}\newpage
%       \textcolor{headcol}{\begin{center}\Huge\bfseries
%       \thechapter #1\end{center}}\addcontentsline{toc}{chapter}{#1}
% }
\renewcommand{\section}[1]{\refstepcounter{section}\newpage
      \textcolor{headcol}{
\centerline{\huge\bfseries\thesection~#1}\\[0.5ex]}\addcontentsline{toc}{section}{#1}}
%\begin{center}\huge\bfseries\thesection~#1\end{center}}\addcontentsline{toc}{section}{#1}}
\renewcommand{\subsection}[1]{\newpage 
      \textcolor{headcol}{\centerline{\parbox{\textwidth}{\LARGE\bfseries #1}}}}
\renewcommand{\subsubsection}[1]{\newpage
      \textcolor{headcol}{\centerline{\parbox{\textwidth}{\Large\bfseries #1}}}}
\renewcommand{\aufg}%%%%%%%%%%%% Aufgabe %%%%%%%%%%%%
        {\refstepcounter{AufgNr}
         \par {\bfseries\large Aufgabe \arabic{AufgNr}\\[0.5ex]}%
         \setcounter{TeilaufgNr}{1}%
        }


\newcommand{\nopb}[1]{\enlargethispage{#1}}

\renewcommand{\small}{\large}
\renewcommand{\footnotesize}{\normalsize}

%%% Local Variables: 
%%% mode: latex
%%% TeX-master: t
%%% TeX-master: t
%%% End: 
